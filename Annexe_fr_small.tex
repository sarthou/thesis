\chapter{Résumé en Français}
\label{app:fr_small}

Nous fournissons ici un résumé en langue française des travaux présentés dans ce manuscrit de thèse.

\textbf{Résumé :}
%%%%%%%%%%%%%%%%%%%%%% ATTENTION ! SI MODIFICATION => MODIF SUR ADUM AUSSI !!!
En tendant à intégrer des robots dans des environnements de notre vie quotidienne, le besoin d'une représentation des connaissances avancés et de raisonnements associés ne cesse d'augmenter, dans le but de permettre aux robots de comprendre les elements qui composent ces environements. En considérant la présence d'humains dans de tels environnements, et donc la nécessité d'interagir avec eux, cela exacerbe ce besoin de connaissances avec des exigences supplémentaires. Les connaissances ne sont plus utilisées par le robot dans le seul but d'agir physiquement sur son environnement mais aussi dans le but d'être partagées avec les humains, par la communication. Par conséquent, la connaissance ne doit plus être uniquement compréhensible par le robot lui-même mais doit aussi pouvoir être exprimée. % 803

Dans la première partie de cette thèse est présenté le logiciel Ontologenius. C'est un système permettant de maintenir des bases de connaissances sous forme d'ontologie, de raisonner dessus et de les gérer dynamiquement. Nous commençons par expliquer en quoi ce logiciel est adapté aux applications d'interaction homme-robot (HRI), avec la possibilité de représenter la base de connaissances du robot ainsi qu'une estimation des bases de connaissances des partenaires humains pour mettre en œuvre la théorie de l'esprit. Nous poursuivons avec une présentation de ses interfaces pour l'utiliser dans des algorithmes visant à développer des fonctionnalités cognitives. Cette partie se termine par une analyse des performances, démontrant son utilisabilité en ligne. % 1566

Dans une deuxième partie, cette thèse présente deux problèmes d'exploration des connaissances autour du thème général du référencement spatial et de l'utilisation des connaissances sémantiques. Il commence par la description d'un itinéraire visant à proposer un ensemble d'itinéraires possibles menant à une destination cible. Pour ce faire, nous proposons une ontologie permettant de décrire la topologie d'environnements intérieurs et deux algorithmes de recherche d'itinéraires. Le deuxième problème d'exploration des connaissances abordé dans cette thèse est le problème de génération d'expressions référentes (REG). Il vise à sélectionner l'ensemble optimal d'informations à communiquer afin de permettre à un auditeur d'identifier l'entité référencée dans un context donné. Ce dernier algorithme est ensuite affiné pour utiliser les activités passées provenant d'une action conjointe entre un robot et un humain, afin de générer de nouveelles sortes d'expressions référentes. Il est également intégré à un planificateur de tâches symbolique pour estimer la faisabilité et le coût des futures communications. % 2679

Cette thèse se termine par la présentation de deux architectures cognitives, la première utilisant la contribution de description d'itinéraire et la seconde utilisant les contributions autour de la Génération d'Expression Référente. Les deux utilisent Ontologenius pour gérer la base de connaissances sémantique. À travers les deux architectures, nous présentons comment cette base de connaissances a progressivement pris un rôle central, fournissant des connaissances à tous les composants des architectures.


\section*{Introduction}

Dans cette thèse, nous présentons plusieurs contributions autour de l'utilisation d'ontologie comme moyen de représenter les connaissances pour le robot ainsi que pour représenter une estimation des connaissances des partenaires du robot. Une ontologie est un graphe de connaissances avec en plus une spécification formelle et explicite de la signification partagée des concepts qui y sont utilisés. En robotique, l'utilisation d'une telle représentation formelle et explicite permet une unification des connaissances au sein d'une architecture entière. La connaissance n'est plus omniprésente parmi les composants, chacun possédant la partie dont il a besoin sans accord. Avec une ontologie, la connaissance devient une ressource partagée. Cette notion de connaissance partagée est également importante lorsqu'on parle de systèmes multi-robots. L'utilisation d'une ontologie permet à tous les robots de communiquer en utilisant le même vocabulaire. Par conséquent, il peut faciliter leur interaction, même s'ils ne reposent pas sur la même architecture.

En étendant les applications multi-robots, nous atteignons les applications multi-agents et par conséquent les applications pour l'interaction homme robot. La connaissance qu'une ontologie représente est la connaissance de la façon dont nous, en tant qu'êtres humains, nous percevons notre environnement et comment nous l'interprétons. Certes une ontologie est compréhensible par la machine mais surtout, à la base, compréhensible par l'homme, à la différence des réseaux de neurones par exemple. Grâce à toutes ces caractéristiques et parce qu'elle vient de la connaissance humaine, l'ontologie peut convenir pour représenter une estimation de la connaissance humaine pour un robot. De plus, en essayant de l'aligner sur le savoir humain, de la même manière que plusieurs robots peuvent utiliser le vocabulaire qu'une ontologie contient pour communiquer, un robot pourrait l'utiliser pour communiquer avec un humain, que ce soit pour interpréter un acte de communication ou pour en produire un. 

\section*{Ontologenius : une mémoire sémantique pour l'interaction Homme-Robot}

Dans ce chapitre, nous présentons Ontologenius, un logiciel open source et léger pour maintenir un graphe de connaissances à l'aide d'une ontologie. Ce logiciel avait été développé pour des applications d'Interaction Homme-Robot avec la possibilité de maintenir plusieurs bases de connaissances à la fois, l'une représentant les connaissances du robot et les autres étant l'estimation des connaissances des partenaires du robot. De plus, concernant la gestion de plusieurs instances à la fois, Ontologenious est livré avec une fonctionnalité de copie profonde pour récupérer l'état d'une base de connaissances à un moment donné, puis pour la modifier librement. Pour représenter plusieurs états de connaissance d'un même agent et éviter la création de trop d'instances qui pourraient ralentir le CPU, avec Ontologenius nous avons proposé une sorte de système de versionnage, gardant une trace des changements. Concernant la récupération de connaissances, avec Ontologenius nous avons choisis de fournir un ensemble de requêtes précises travaillant au niveau sémantique mais permettant l'exploration de la structure de la connaissance plutôt que simplement les relations entre entités.

\section*{Recherche d'un itinéraire avec des connaissances sémantiques}

Sur la base d'Ontologenius, nous présentons plusieurs contributions utilisant les connaissances stockées. Dans ce chapitre, nous proposons une manière de décrire la topologie des environnements intérieurs à l'aide d'une ontologie. La représentation résultante a été appelée la représentation spatiale sémantique. Dans le cadre d'une tâche de description d'itinéraire et en utilisant la représentation spatiale sémantique, nous présentons une combinaison de deux algorithmes capables de trouver plusieurs itinéraires menant à une destination. En utilisant la même représentation, nous présentons ensuite un troisième algorithme pour générer l'explication de l'itinéraire, sous forme de phrase, en respectant les trois bonnes pratiques identifiées par Allen dans~\cite{allen_2000_principles}. La façon dont l'explication est générée permet à l'humain guidé d'effectuer une visite imaginaire de l'environnement, augmentant les chances d'atteindre la destination demandée.

\section*{Génération d'expression de référence basée sur une ontologie}

Poursuivant sur l'exploitation des connaissances et plus précisément sur la communication spatiale, nous présentons maintenant une contribution autour de la tâche de génération d'expressions référentes. Comme expliqué dans~\cite{reiter_2000_building}, il s'agit de ``comment produire une description d'une entité qui permet à l'auditeur d'identifier cette entité dans un contexte donné''. Alors que cette tâche est étudiée depuis des décennies, aucune des méthodes existantes n'a tenté d'utiliser une ontologie comme base de connaissances, travaillant la plupart du temps sur une base de connaissances dédiée à la tâche. De plus, parce que leur base de connaissances est dédiée à la tâche, elle n'est composée que de relations utilisable pour communiquer avec un humain. Cependant, compte tenu d'une base de connaissances partagée dans l'architecture, certaines connaissances peuvent avoir un usage purement technique pour le fonctionnement du robot. Plus important encore, aucun des travaux existants n'a vraiment pris en compte la notion de contexte. Leur seul but était de désigner une entité dans un état donné d'un environnement. Cependant, dans son utilisation pour de l'interaction homme-robot, la génération d'expression de référence n'est qu'une action d'une tâche plus large, dans laquelle un contexte existe, qu'il s'agit d'informations déjà connues sur l'entité à laquelle se référer ou d'une restriction implicite sur les entités concernées. Nous présentons ainsi un nouvel algorithme gérant toutes les problématiques précédemment évoquées et utilisant une ontologie comme base de connaissances. De plus, à travers des comparaisons avec d'autres algorithmes, nous montrons que notre contribution est, à ce jour, la plus efficace, résolvant de la plupart des problèmes en moins d'une milliseconde. Enfin, comme une preuve d'utilisabilité, l'algorithme a été intégré dans une architecture robotique avec une base de connaissances mise à jour en permanence par la perception. 

\section*{Estimation de la faisabilité et du coût de la communication lors de la planification des tâches}

Profitant de la haute performance de notre algorithme de génération d'expressions référentes, dans ce chapitre, nous proposons une méthode l'intégrant à un planificateur de tâches. Le but de cette méthode est de doter le planificateur de tâches de la capacité d'estimer la faisabilité et le coût de la communication, d'éviter les impasses et de trouver des plans minimisant la complexité de la communication. Certes, dans notre contribution, la communication s'est limitée à la référence d'entités mais pourrrait s'étendre à d'autres. Premièrement, le planificateur de tâches que nous utilisons est HATP, un planificateur de tâches prenant en compte l'humain, ce qui signifie qu'il est capable de planifier pour le robot et son partenaire, en estimant leur état mental futur et leurs capacités à assigner des tâches à l'un ou l'autre. Pour estimer les futurs états mentaux, HATP dispose d'une représentation interne dédiée, limitée aux entités utilisées dans la tâche à planifier. Cependant, pour fonctionner, notre algorithme de génération d'expressions référentes a besoin d'une ontologie comme base de connaissances et doit prendre en compte toutes les entités de l'environnement. Pour le faire fonctionner, nous présentons donc un schéma permettant au planificateur de mettre à jour une ontologie, représentant la future connaissance estimée du partenaire humain, afin de pouvoir exécuter l'algorithme dessus. Cette contribution tire pleinement avantage des fonctionnalités apportées par ontologenius, à savoir la possibilité de maintenir une instance par agent, et de copier une instance à un moment donné pour la modifier librement. La méthode résultante est implémentée dans une architecture robotique en tant que preuve de concept. 

\section*{Étendre le REG avec des connaissances sur les activités passées}

Avec l'intégration de l'algorithme de génération d'expressions référentes avec un planificateur de tâches, nous mettons en évidence le fait que l'acte de faire référence à une entité, apparaît la plupart du temps dans le contexte d'une tâche. Au cours de cette tâche, les agents agissent avec l'environnement et manipulent les entités de l'environnement. Sur cette base, dans ce chapitre, nous proposons d'utiliser cette connaissance supplémentaire sur les activités des agents avec les entités de l'environnement comme une nouvelle information, utilisable pour s'y référer. En continuant à utiliser HATP pour le planificateur de tâches, nous proposons d'abord un moyen de représenter le domaine de plannification de tâches dans une ontologie, ainsi que l'exécution de ce plan, que nous avons appelé la trace d'exécution hiérarchique. Ensuite, nous présentons les modifications apportées à notre algorithme de génération d'expression de référence d'origine, lui permettant d'utiliser la représentation des activités passées pour générer un nouveau type d'expression de référence. 

\section*{Aller plus loin que les relations binaires dans le REG}

Même si l'adaptation de l'algorithme de génération d'expressions référentes du chapitre précédent apporte de nouvelles possibilités, nous montrons dans ce chapitre qu'il présente certaines limites. La principale est la connaissance a priori dont il a besoin sur la représentation. Elle est ainsi restreinte à une représentation unique des activités passées. Cependant, nous avons vu dans la littérature qu'il existe de nombreuses représentations d'activités, chacune créée pour des applications particulières. Dans ce chapitre, nous explorons donc des modèles d'ontologie communs utilisés pour représenter de telles relations complexes sous la forme de relations n-aires. En leur ajoutant un ensemble de motifs paramétriques simples en tant qu'étiquettes, nous avons créé ce que nous avons appelé des relations composées. Avec ce nouveau type de relation, nous adaptons notre algorithme de génération d'expression de référence original pour fournir un moyen plus générique de générer une expression de référence. L'algorithme résultant fonctionne toujours avec une description des activités passées, mais aussi avec toute représentation utilisant une relation n-aire. De plus, nous montrons que les performances du nouvel algorithme sont meilleures que celles de l'algorithme précédent et égales à celles de l'algorithme d'origine lorsqu'aucune relation composée n'est utilisée dans la représentation.

\section*{Un robot dans le centre commercial : le projet MuMMER}

A travers les chapitres suivants, nous présentons deux architectures robotiques impliquant les contributions présentées tout au long de cette thèse. Dans le chapitre courant, nous présentons le projet MuMMER et l'architecture qui en résulte. Le but du projet MuMMER était de développer une architecture robotique permettant à un robot de guider un client dans un centre commercial et de discuter avec lui. Par guider, nous n'entendons pas accompagner le client mais plutôt lui expliquer le parcours à suivre. Pour pouvoir trouver l'itinéraire à expliquer et générer la phrase d'explication, nous utilisons la contribution sur la description de l'itinéraire. Pour conserver la représentation de l'environnement et la partager avec d'autres composants, nous utilisons Ontologenius. L'architecture résultante, incarnée dans un robot Pepper, a fonctionné pendant 14 semaines dans un centre commercial en Finlande. 

\section*{La tâche du directeur : Évaluer les architectures cognitives}

Pour terminer cette thèse, dans ce chapitre, nous présentons une intégration de l'algorithme de génération d'expressions référentes dans une architecture robotique complète dédiée aux applications d'Interaction Homme-Robot. De plus, avec cette nouvelle architecture, nous montrons comment nous plaçons la connaissance au centre de l'architecture, utilisée par tous les composants. De plus, dans ce chapitre, nous introduisons une nouvelle tâche, inspirée d'expériences de psychologie, pour évaluer l'architecture cognitive de l'interaction Homme-Robot. Partant de cette tâche et eu égard à l'architecture mise en œuvre, nous terminons ce chapitre par un ensemble de défis que nous pensons être intéressants pour la communauté. 