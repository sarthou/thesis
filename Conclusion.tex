
\chapter*{Conclusion: represententing, storing, exploring, communicating}
\addstarredchapter{Conclusion: represententing, storing, exploring, communicating} %Sinon cela n'apparait pas dans la table des matières
\markboth{Conclusion}{}

%\ifdefined\included
%\else
%\bibliographystyle{acm}
%\bibliography{These-refs}
%\end{document}
%\fi

In this thesis, we presented several contributions around the use of ontology as a way to represent knowledge for the robot as well as to represent an estimation of the robot's partners knowledge. An ontology is a knowledge graph with on top a formal and explicit specification of the shared meaning of the concepts used in it. In robotic, the use of such a formal and explicit representation provides a unification of the knowledge among an entire architecture. The knowledge is no more ubiquitous among the components, each owning the part it needs without agreement. It became a shared resource. This notion of shared knowledge is also important is with speak about multi-robot systems. Using an ontology allows all the robots to communicate using the same vocabulary. Consequently, it can facilitate their interaction, even if they do not rely on the same architecture.

Extending multi-robot applications, we reach multi-agent applications and consequently \acrfull{hri}. The knowledge an ontology represents is the knowledge of how we, as humans, perceive our environment and how we interpret it. For sure an ontology is machine-understandable but especially, at the base, human understandable, at the difference of neural networks for example. Thanks to all these characteristics and because it comes from the human, ontology can be suitable for a to represent an estimation of the human knowledge. In addition, trying to align it with the human knowledge, as robots can use the vocabulary it maintains to communicate, a robot could use it to communicate with a human, that it is to interpret a communication act or to produce one.

In chapter \ref{chap:2}, we presented Ontologenius, an open-source and lightweight software to maintain knowledge graph using ontology. This software had been developed for \acrshort{hri} application with the ability to maintain several \acrlong{kb} at the time, one representing the robot's knowledge and the others being the estimation of the robot's partners knowledge. In addition, regarding the management of several instances at the time, Ontologenious comes with a deep-copy feature to catch the state of a \acrshort{kb} at a given moment, then to modify it freely. To represent several knowledge states of the same agent and to avoid the creation of too many instances that could slow down the CPU, with Ontologenius we proposed a kind of versioning system, keeping a trace of the changes. Regarding the knowledge retrieve, with Ontologenius we choose to provide a set of precise query working at the semantic level but allowing the exploration of the structure of the knowledge rather than simply the relations between entities.

On the basis of Ontologenius, we have presented several contributions making use of the stored knowledge. In chapter \ref{chap:3}, we proposed a way to describe the topology of indoor environments using an ontology. The resulting representation was called the \acrfull{ssr}. In the context of a route description task and using the \acrshort{ssr}, we presented a combination of two algorithms able to find several routes leading to a destination. Using the same representation, we presented a third algorithm to generate the route explanation, in the form of a sentence, by respecting the three good practices identified by Allen in~\cite{allen_2000_principles}. The way the explanation is generated allows the guided human to perform an imaginary tour of the environment, increasing the chances to reach the requested destination.

Continuing on the knowledge exploitation and more precisely on spatial communication, from chapter \ref{chap:4} to \ref{chap:7}, we presented contributions around the \acrfull{reg} task. As explained in~\cite{reiter_2000_building}, it is the concern of "how we produce a description of an entity that enables the hearer to identify that entity in a given context". While this task has been studied for decades, none of the existing methods had tried to use an ontology as \acrshort{kb}, working most of the time on \acrshort{kb} dedicated to the task. In addition, because their \acrshort{kb} are dedicated to the task, it is only composed of relation usable to communicate with a human. However, considering a shared \acrshort{kb} among the architecture, some knowledge can have a purely technical use for the robot functioning. More importantly, none of the existing works really considered the notion of context. Their only goal was to designate an entity in a given state of an environment. However, in \acrshort{hri} use, the \acrshort{reg} is just an action of a wider task in which a context exists, that it is already known information about the entity to refer to or implicit restriction about the entities in concern. In chapter \ref{chap:4}, we thus presented a new algorithm managing all the previously evoked issues and using an ontology as \acrshort{kb}. In addition, through comparisons with other algorithms, we showed that our contribution is, to the date, the most efficient, solving most of the problems is less than a millisecond. Finally, for usability proof, the algorithm has been integrated into a robotic architecture with a \acrshort{kb} continuously update through perception.

Taking advantage of the high performance of our \acrshort{reg} algorithm, in chapter \ref{chap:5}, we proposed a method integrating it with a task planner. The goal of this method was to endow the task planner with the ability to estimate the feasibility and the cost of communication, to avoid deadlock and to find plans minimizing the communication complexity. For sure, in our contribution, the communication was limited to the reference of entities but could be extended to others. First, the used task planner, was \acrshort{hatp}, a human-aware task planner, meaning that it is able to plan for the robot and its partner, estimating their future mental state and their abilities to assigned tasks to one or another. To estimate the future mental states, \acrshort{hatp} has a dedicated internal representation, limited to the entities used in the task to plan. However, to work, our \acrshort{reg} algorithm needs an ontology as \acrshort{kb} and need to take into account all the entities of the environment. To make it work, we thus present a scheme allowing the planner to update an ontology, representing the future estimated knowledge of the human partner, in order to be able to run the algorithm. This contribution fully takes advantage of the features brought by ontologenius, that is was the ability to maintain an instance per agent, and to copy an instance at a given moment to freely modify it. The resulting method was implemented into a robotic architecture as a proof of concept.
