\usepackage[colorinlistoftodos,prependcaption,textsize=tiny]{todonotes}
\usepackage{xargs}   % Use more than one optional parameter in a new commands
\usepackage{xcolor}  % Coloured text
\usepackage{xspace}
\usepackage{hyperref}

\usepackage{amsmath}
\usepackage{amssymb}
\usepackage{bbm}
%\usepackage{mathbbol}

\usepackage{longtable}
\usepackage{booktabs}
\usepackage{courier}

\definecolor{lightgray}{gray}{0.5}

\usepackage{listings}
\usepackage[normalem]{ulem}
\useunder{\uline}{\ul}{}

\newtheorem{theorem}{Theorem}

\usepackage{algorithm}
\usepackage[noend]{algpseudocode} 
% limit indentation in algorithms
\algrenewcommand\algorithmicindent{1.0em}%
\algnewcommand{\IIf}[1]{\State\algorithmicif\ #1\ \algorithmicthen}
\algnewcommand{\EndIIf}{\unskip\ \algorithmicend\ \algorithmicif}

\newcommand*{\field}[1]{\mathbb{#1}}%

%%%%%%%%%%%%%%%%%%%%%%%%
%	 listing formats   %
%%%%%%%%%%%%%%%%%%%%%%%%

\lstdefinestyle{HatpPlan}{
  language=C,
  commentstyle=\itshape\color{green!25!black},
}

\lstdefinestyle{OwlTurtle}{
  language=C,
  keywordstyle=\bfseries\color{darkgray},
  morekeywords={rdf:type, rdfs:domain, rdfs:subPropertyOf, rdfs:range, :hasSubtask, :DecompositionUsedBy, rdfs:subClassOf, :hasDecomposition, owl:inverseOf},
  alsoletter=:
}

\lstdefinestyle{OwlTurtle_indiv}{
  language=C,
  keywordstyle=\bfseries\color{darkgray},
  morekeywords={rdf, rdfs, type, domain, subPropertyOf, range, hasSubtask, DecompositionUsedBy, subClassOf, hasDecomposition, Cut_hasParameter, A, V, K, hasColor, hasCadModel, isOn, isInRegion, isAtEndOfPath, isAtLeftOfPath, hasAtRight, isInFrontOf}
}

\lstdefinestyle{Labels}{
  language=C,
  basicstyle = \fontencoding{T1}\ttfamily \color{black} \footnotesize
}

%%%%%%%%%%%%%%%%%%%%%%%%
%	  TODO commands    %
%%%%%%%%%%%%%%%%%%%%%%%%

\setlength{\marginparwidth}{3cm}
\newcommandx{\unsure}[2][1=]{\todo[linecolor=red,backgroundcolor=red!25,bordercolor=red,#1]{#2}}
\newcommandx{\change}[2][1=]{\todo[linecolor=blue,backgroundcolor=blue!25,bordercolor=blue,#1]{#2}}
\newcommandx{\info}[2][1=]{\todo[linecolor=olive,backgroundcolor=olive!25,bordercolor=olive,#1]{#2}}
\newcommandx{\improvement}[2][1=]{\todo[linecolor=violet,backgroundcolor=violet!25,bordercolor=violet,#1]{#2}}
\newcommandx{\thiswillnotshow}[2][1=]{\todo[disable,#1]{#2}}

%%%%%%%%%%%%%%%%%%%%%%%%
%	Knowledge bases    %
%%%%%%%%%%%%%%%%%%%%%%%%

\newcommand \kb{K}

% Ontology definition

\newcommand \kbs{\kb_S}
\newcommand \Abox{\mathbb{A}}
\newcommand \Tbox{\mathbb{T}}
\newcommand \Rbox{\mathbb{R}}

\newcommand \propset{P}
\newcommand \classset{T}
\newcommand \indivset{A}
\newcommand \relationset{R}
\newcommand \annotationset{E}

\newcommand \domainset{Dom}
\newcommand \rangeset{Ran}
\newcommand \inclset{Incl}
\newcommand \invset{Inv}
\newcommand \inheritset{C}

\newcommand \indiv{a}
\newcommand \class{t}

\newcommand \labelfunc {\mathcal{L}}
\newcommand \alabel{\mathcal{L}_a}
\newcommand \tlabel{\mathcal{L}_t}
\newcommand \plabel{\mathcal{L}_p}
\newcommand \alabelag[1]{\mathcal{L}_{a#1}}
\newcommand \tlabelag[1]{\mathcal{L}_{t#1}}
\newcommand \plabelag[1]{\mathcal{L}_{p#1}}

\newcommand \subject{s}
\newcommand \property{p}
\newcommand \object{o}
\newcommand \relation{r}

\newcommand{\sparql}{\textsc{sparql}}

% Timeline definition

\newcommand \kbe{\kb_E}
\newcommand \tinterval{\mathcal{T}}
\newcommand \tstart{\tau_s}
\newcommand \tend{\tau_e}
\newcommand \taction{\alpha}

%%%%%%%%%%%%%%%%%%%%%%%%
%	      REG          %
%%%%%%%%%%%%%%%%%%%%%%%%

\newcommand \goalindiv{\indiv_t}
\newcommand \usablepropset{U}
\newcommand \problem{\mathcal{P}}
\newcommand \solution{\mathcal{S}}

\newcommand \costfunc{\mathcal{C}}
\newcommand \varset{X}
\newcommand \var{x}

\newcommand \cost{\mathbbm{c}}
\newcommand \node{\mathbbm{n}}
\newcommand \state{\mathbbm{s}}
\newcommand \action{\mathbbm{a}}

\newcommand \softdiff{\,\delta\,}
\newcommand \harddiff{\,\Delta\,}

\newcommand \symboltable{\mathcal{S}}
\newcommand \matchtable{\mathcal{M}}

\newcommand \us{$\mu$s\xspace}

\newcommand{\tovariable}{\textsc{ToVariable}}
\newcommand{\toquery}{\textsc{ToQuery}}
\newcommand{\sparqlresult}{\textsc{SparqlResult}}
\newcommand{\createchild}{\textsc{CreateChild}}
\newcommand{\actions}{\textsc{Actions}}
\newcommand{\goaltest}{\textsc{GoalTest}}
\newcommand{\typingactions}{\textsc{TypingActions}}
\newcommand{\differenceactions}{\textsc{DifferenceActions}}

\newcommand{\additions}{\textsc{CreateAddition}}
\newcommand{\typingadditions}{\textsc{TypingAdditions}}
\newcommand{\differenceadditions}{\textsc{DifferenceAdditions}}
\newcommand{\actingadditions}{\textsc{ActingAdditions}}
\newcommand{\completionadditions}{\textsc{CompletionAdditions}}
\newcommand{\compoundadditions}{\textsc{CompoundAdditions}}
\newcommand{\createtree}{\textsc{CreateTree}}
\newcommand{\getsubtrees}{\textsc{getSubTrees}}

%%%%%%%%%%%%%%%%%%%%%%%%
%	      HTN          %
%%%%%%%%%%%%%%%%%%%%%%%%

\newcommand \tasknetwork{N}
\newcommand \task{n}
\newcommand \primtaskset{Pt}
\newcommand \abstaskset{At}
\newcommand \decomposet{D}
\newcommand \decompo{d}
\newcommand \abstracttask{\upsilon}
\newcommand \HTN{Pl}