\ifdefined\included
\else
\setcounter{chapter}{3} %% Numéro du chapitre précédent ;)
\dominitoc
\faketableofcontents
\fi

\chapter{Searching for a route with semantic knowledge}
\chaptermark{Searching for a route with an ontology}
\label{chap:3}
\minitoc

The contribution presented in this chapter is excerpted from our work, published in the proceedings of the Spatial Language Understanding (SpLU) 2019 workshop~\cite{sarthou_2019_semantic}. In this manuscript, the contribution is more detailed and discussed. This work is part of the MuMMER project, aiming at developing a robot guide in a mall. At the end of this thesis, a chapter is dedicated to the presentation of the project and the integration of the current contribution is a robotic system.

\section{Introduction}

We all have already been requested, or have ourselves request, for a route toward public space in a city, a shop in a shopping centre, or more simply a toilet in a house. When providing such information to a lost person we perform what is commonly called a guidance task. Even if it can seem evident for us, developing a robot able to perform it can be challenging. In this chapter, we choose to focus on the sub-task consisting to generate the explanation sentence. This sub-task is called the route description. To perform it, we first need a set of knowledge about the environment in which the guided person will walk, such as the paths, the intersections of the paths, or the elements alongside them. Then, we need a set of "good practices" to provide a route easy enough to follow and to remember.

In the Human-Robot Interaction (HRI) field, robots guides have been study intensively and deployed into shopping centers~\cite{okuno_2009_providing}, museums~\cite{burgard_1999_museum, clodic_2006_rackham, siegwart_2003_robox}, or airport~\cite{triebel_2016_spencer}. From a knowledge representation point of view, we can notice the use of metrical representations~\cite{thrun_2007_simultaneous} or topological representations~\cite{morales_2011_modeling} to represent the environment in which the robot evolves. Since we focus on the route description task, we consider that the robot does not accompany the human to his final destination but rather explains how to reach it. Consequently, the metrical representation will not be considered as being mainly used for navigation purpose~\cite{thrun_2007_simultaneous}. To perform more specifically a route description, topological knowledge is not sufficient. In addition to the topology of the environment, the robot needs to know the types of the elements composing the environment and their names in natural language. Some contributions have thus try to mix metrical or topological representations with semantic ones to hold this additional knowledge~\cite {satake_2015_should, chrastil_2014_cognitive, zender_2008_conceptual}. However, mixing them can create a lack of uniformity among the overall knowledge representation. In this way, creating a unique representation allowing a robot to compute routes and expressing them could ensure uniformity among the knowledge.

Even having a rich enough representation of its environment, the robot has to find a route not for it but for the guided human. A robot accompanying the human only has to determine a path, adapted to its capacities and interpretable only by it. Providing a route to a human, the route has to be adapted to the human capabilities. For example, in an outdoor environment, we will not give the same route for a car driver or a cyclist. In the context of a mall, we will not give a route with stairs along to a mobility-impaired person or to someone with a shopping cart. Once an adapted route computed, the robot has to explain it. Where interactive maps only have to highlight a path, here, the robot has to generate a sentence that the human will memorize. For sure the robot will not instruct a human with a sentence like "walk 30 meters them turn -90 degrees". This would not be adapted. The use of orientation and reference to elements of the environment will be needed through a sentence like "walk until the florist then turn left".

The first contribution of this chapter is a \textbf{unified representation} of an indoor environment using an ontology, to include both topological and semantic knowledge. Then, on the basis of this representation, we propose a first algorithm to \textbf{find a suitable route} to be explained to a human and a second algorithm to \textbf{verbalize a route} in an appropriate way.

First, we review the literature concerning semantic representation of indoor environment and route description. Then, we introduce the reader to our unified semantic representation under the name of Semantic Spatial Representation (SSR). We then present the algorithm used to compute the route and in a second time the algorithm to verbalize the previously computed route. We end this chapter with experimental results on both emulated and real environments.

\section{Related work}

\subsection{Describing a route}

In the literature, a route description task is defined as being a particular kind of spatial description. First, from a cognitivist point of view, Denis in~\cite{denis_1997_description} has identified three main cognitive operations used to generate such a spatial discourse: 1) the activation of an internal representation of the environment, 2) the planning of a route in this representation, 3) and finally the formulation of the procedure to follow. From a computer science point of view, Cassell in~\cite{cassell_2007_trading} view the second operation as the fact of finding a set of routes segment, each connecting two important points, and the third operation as chronologically explaining the route segments. In the same way, Mallot in~\cite{mallot_2009_embodied}, see the second operation as the fact of selecting a sequence of places leading to the objective, and the third as managing declarative knowledge to choose the right action to explain at each point of the sequence. While both second operations are equivalent, the thirds about the formulation of the procedure are rather complementary.

The route description task has been extensively studied through verbal and textual communication to understand how humans communicate spatial knowledge. The goal of such studies has been to identify the invariants but also the good practices ensuring the success of the task. Through five experiments in both urban and interior environments, Allen is~\cite{allen_2000_principles} has identified three basic practices seen as being important for communicating knowledge about routes. They can be summarized as follows: a) respect the spatiotemporal order, b) concentrate on the information about the points of choice and c) use landmarks that the listener can easily identify.

This latter practice about the use of landmarks, also called reference marks, has been identified by Tversky in~\cite{tversky_1999_pictorial} has been critical information for the success of a route description. With his anterior contribution~\cite{tversky_1998_space}, he finds that in addition to information about actions, reorientation, and direction, 91\% of the guidance instructions contains the use of landmarks. These results trend at confirming the ones of Denis in~\cite{denis_1997_description}. In an anterior study, Montello in~\cite{montello_1993_scale} tries to identify when the use of landmarks appear in a description. Defining the \textit{Vista} space as being the area within sight and the \textit{Environmental} as being the rest of the environment reachable through locomotion, he finds that guides usually used landmarks when the target places were no longer in the \textit{Vista} space but in the \textit{Environmental} one. Moreover, with regard to~\cite{tversky_1999_pictorial}, the use of landmark more precisely appears during an explanation of a direction changing. In addition, their choice is based on salient features over a route description~\cite{nothegger_2004_selection}.

\begin{figure}[ht!]
\centering
\includegraphics[scale=0.22]{figures/chapter3/landscape/landscape.png}
\caption{\label{fig:chap3_shortest} Comparison of two routes in terms of complexity and length. Even if the blue route (. . .) is the shortest many directions changing are required. Each of them is a risk for the guided person to make a mistake and be lost again. The red route (- - -), although being a bit longer, is easier to explain and to remember, and has few directions changing.}
\end{figure}

Even if the use of landmarks helps at understanding direction changing by anchoring the action to be performed, they still are a risk for the guided person to make a mistake, taking the wrong path. Where the length of the route would be an important criterion is the choice of a route, its complexity is also to be taken into account when we need to explain it. Morales in~\cite{morales_2015_building} argued that reducing the route complexity, in terms of the number of stages composing it, should be prefered to its length. This feature reduces the risk of mistake concerning the choice to make along it and also has an impact on its understanding and memorization. This criteria of minimal explanation can be compared to the Grice's Maxim of quality~\cite{grice_1975_logic}. In the example of figure~\ref{fig:chap3_shortest}, some should prefer to explain the red route rather than the blue one, even if it is longer.

Finally, to explain the same route Taylor in~\cite{taylor_1992_spatial} has noticed that a speaker can use two kinds of perspective. First, the \textit{survey} perspective trend at adopting a bird's eye view point of the environment, meaning a top view of it like as looking at a map. With this perspective, the speaker refers to the different landmarks of the route with respect to one another. They are thus referred to using terms including north-south-east-west. This perspective is opposite to the \textit{route} perspective. With such a perspective, the speaker mentally navigates along the route, making an imaginary tour of the environment. As a result, he refers to the landmarks with respect to the future guided person position along the route. The landmarks are thus referred to using terms like left, right, front, or back. In~\cite{taylor_1996_perspective}, they notice that the survey perspective is generally used for open environments whereas the route perspective is generally used in environments with already identified paths. For indoor environments, the latter should thus be preferred to facilitate route understanding and memorization.

\subsection{Environment represention to compute routes}

Regarding the environment representation generally used to find itineraries, we can first take a look to GNSS road navigation systems. In \cite{liu_1997_route} or \cite{cao_2009_gps}, we find the same principle of a topological network representing the roads with semantic information attached to each of them. Such representation seems adapted regarding the performance required for such systems operating in very large areas. However, GNSS road navigation systems must respond only to this unique task of finding a path when a robot is expected to be able to answer to various tasks. For our application, we thus need a representation that can be used more widely while still allowing the search for routes.

%Morales et al. \cite{morales_building_2015} indicate that naming parts of a geometric map does not leave the opportunity to compute such perspective. As in \cite{satake_field_2015}, we have chosen to develop our representation with an ontology as it allows to reason about the meaning of words and thus improve the understanding of human demands. In addition, we propose a way to merge the topological representation into the semantic representation (the ontology) to get the meaning of the environment elements while keeping a description of the connectivity of the elements of the environment. We propose to name it semantic spatial representation (SSR). 

% More than the extension of the spatial semantic hierarchy (SSH) \cite{kuipers_spatial_2000} allowing the representation of the environment

%This paper focuses on the presentation of the SSR and on its usability for the route description task. For now, all the ontologies used to test the SSR have been made by hand. However, many recent research work leads to automatically generate a topological representation of an environment from geometric measurements (e.g. Region Adjacency Graphs \cite{kuipers_local_2004}, Cell and Portal Graphs \cite{lefebvre_automatic_2003} or hierarchical models \cite{lorenz_hybrid_2006}, or from natural language \cite{hemachandra_learning_2014}). We have not done it yet, but our system could benefit from this work to generate a representation of an environment using SSR, which would solve the complexity of creating such a representation by hand.

\section{The Semantic Spatial Representation}

\subsection{The SSR classes}

\begin{figure}[ht!]
\centering
\includegraphics[scale=0.4]{figures/chapter3/ssr_tbox.png}
\caption{\label{fig:chap3_tbox} Representation fo the TBox (classes hierarchy) of the Semantic Spatial Representation used to describe the topology of an indoor environment. While the top part is inherent to the SSR, the bottom one extends the latter to provide more granularity.}
\end{figure}

\subsection{The SSR properties}

\begin{figure}[ht!]
\centering
\includegraphics[scale=0.4]{figures/chapter3/ssr_rbox.png}
\caption{\label{fig:chap3_rbox} Representation fo the RBox (properties hierarchy) of the Semantic Spatial Representation used to describe the topology of an indoor environment.}
\end{figure}


\section{Finding routes to the right destination: A two-level search}

\subsection{The region-level: Trim down the search}

\begin{figure}[ht!]
\centering
\includegraphics[scale=0.22]{figures/chapter3/building_regions.png}
\caption{\label{fig:chap3_regions} Representation of an environment at the region-level. Regions are linked trhough interfaces. We know that the starting point of the search is in \textit{region\_1} and the goal place is in \textit{region\_3}. }
\end{figure}


\subsection{The place-level: Refine the search}

\begin{figure}[ht!]
\centering
\includegraphics[scale=0.28]{figures/chapter3/region1.png}
\caption{\label{fig:chap3_region1} Representation the \textit{region\_1} at the place-level. A region is composed of paths (here corridors only) connected through intersections. We know that the starting point of the search is along \textit{corridor\_1} and the local goal place is in \textit{corridor\_5}. }
\end{figure}

\subsection{Selecting the most suitable route}

\section{Genarating an explanation in natural language}

\subsection{Putting the robot in your shoes}

\subsection{A pattern-based generation}

\section{Experiment in emulated and real environment}



