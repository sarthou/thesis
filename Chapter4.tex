\ifdefined\included
\else
\setcounter{chapter}{4} %% Numéro du chapitre précédent ;)
\dominitoc
\faketableofcontents
\fi

\chapter{Ontology-based Referring Expression Generation}
\minitoc

\section{Related work}

Referring Expression Generation is a today classic task in Natural Language Generation \cite{gatt_2018_survey} that has been studied for decades. It has been defined by Reiter as the concern of "how we produce a description of an entity that enables the hearer to identify that entity in a given context"~\cite{reiter_2000_building}. Over time, the criteria for a good Referring Expression (RE) have been refined but still take their roots from the Grice's maxims~\cite{grice_1975_logic}. The maxim of \textit{manner} requires the communication to be unambiguous. It is also the referential success for the target entity to be unambiguously identified by the RE hearer. The maxim of \textit{relation} requires the communication to be relevant regarding the current context both the context of the task to achieve and the current world state. If you are asking someone to give you an object that is in the room where you are, you can reasonably assume that the objects in the rest of the house are not ambiguous with the one you are requesting. The maxim of \textit{quality} seems to be evident and requires the communication to be true. If you are asking a bootle and you do not know if it is full or not, you should not use this information to refer to the bottle. Finally, the maxim of \textit{quantity} requires the communication to be as informative as required but not more informative than required. In simple words, to be brief. In the context of REG, the hearer must understand quickly want you are talking about. Moreover, giving unnecessary information could lead to false implications. Saying "give me the red pen" could imply that at least one other non-red pen exists and such warn the hearer to not do the mistake to take the wrong one. If no other pen exists regarding the current context, the sentence "give me the pen" is thus sufficient.

Dale and Reiter are considered as being the pioneers of the Referring Expression Generation and have proposed over years three main algorithms solving it. Two first two fundamental approaches are the Depth First Search (DFS)~\cite{dale_1989_cooking} and the Full Brevity~\cite{dale_1992_generating}. While the first algorithm does not always find an optimal solution in terms of the number of relations used, the second does it at the cost of an exhaustive search. The most notable advance was thus the Incremental Algorithm (IA) first presented in~\cite{reiter_1992_fast} then refine in~\cite{dale_1995_computational}. With this algorithm, the notion of preference over features has been highlighted. This notion aims at representing the fact that some features are easier to understand than others. For example, the color or the shape of an object is easier the understand than spatial relations. However, the major limitation of the presented algorithms is the used knowledge representation. Because they used a set of attribute-value pairs for each entity, the solutions can only be composed of entity attributes and cannot use relations between entities. To be more precise, the algorithms can give the fact the referred entity is on a table but cannot discriminate the said table among others.

With the introduction of a new representation in the form of a labeled directed multi-graph (also known as the REG graph), Krahmer et al. solved the issue of the reference to other entities~\cite{krahmer_2003_graph}. The related Graph-Based Algorithm (GBA) REG is able to manage relations between entities and, as Dale and Reiter, consider a preference over features. This preference, called Preference Ordering (PO), is represented by a cost assigned to each edge of the graph. The GBA algorithm uses a branch\&bound algorithm which allows finding the optimal RE. On this new basis, extensions have been developed or at least discussed. Regarding the thin link with Description Logic, Krahmer raised the problem of the hierarchy of entity types in~\cite{krahmer_2012_computational}. On its side, Li et al. have proposed an optimized version of the GBA~\cite{li_2017_automatically} GBA allowing an efficiency gain close to 56\%. However, the used task only involved cubes, meaning that their algorithm does not have to take into account the entities' types, which were just added as a post-process. A last interesting GBA is the Longest First (LF) algorithm presented in~\cite{viethen_2013_graphs}. However, more than not respecting the maxim of quantity, its exhaustive search entails poor performance when used on larger realistic knowledge bases.

Learning-based approaches have of course been proposed. The belief network-based method presented in~\cite{yamakata_2004_belief} can only work with objects' attributes. Moreover, the authors indicate that a specific belief network should be constructed and therefore trained for each attribute. Such limitation reduces the genericity of the method. With a log-linear model trained from a corpus of the probability distribution of REs~\cite{fitzgerald_2013_learning}, Fitzgerald et al. face the same problem. Nevertheless, by working on belief bases, Yamakata has highlighted the importance to run the algorithm on the human partner's estimated belief base. It ensures the robot generates a referring solution compatible with concepts estimated to be known by the human.

All the algorithms presented here before are highly dependent on the task to perform. Where learning approaches are dependent on their training corpus, the other relies on knowledge bases integrating only relations usable in the context of the task. Williams et al. proposed a hybrid approach between domain-dependent and domain-independent with a distributed Incremental Algorithm (DIST-PIA)~\cite{williams_2017_referring}. The idea besides this algorithm is to make the core Incremental Algorithm independent of the knowledge representation by making it querying domain-dependent consultants~\cite{williams_2016_framework}. A consultant is an interface of a knowledge base and each knowledge base of the distributed architecture owns one. Each consultant is thus dedicated to a specific set of properties and can be query regarding these properties. To get relations about the location of entities, the Incremental Algorithm can thus query the consultant associated with the knowledge base of locations. While this solution is interesting for distributed architectures, we can ask ourselves about the domain-independence of the core Incremental Algorithm. Indeed, the ordering of the consultants to query in the algorithm can have an impact on the found solution. However, it is worth mentioning that this method has been successfully integrated into a robotic architecture~\cite{williams_2019_dempster}.

At the date, the closest work to the one presented in this chapter is introduced in~\cite{ros_2010_which}. This method uses ontology as a knowledge base. As explained earlier, such knowledge representation is suitable for domain-independent applications. However, here again, the used algorithm takes as a hypothesis that only relations useful for the REG task are present in it. Moreover, in the same way as the IA-based algorithm, their method only supports entities' attributes and not relations between entities. This method has still been integrated into a robotic system that can take advantage of perspective-taking to construct an estimated knowledge base of the human partner to give pertinent RE~\cite{lemaignan_2011_grounding}.

Even if all the presented algorithms rely on different kinds of knowledge representation and have non-equivalent abilities, they all consider a perfect linguistic realisation~\cite{krahmer_2012_computational}. We mean here that they all consider that any concept of their knowledge bases has a word in natural language and can thus be verbalize. Wanting to run on the same knowledge base as the other component of the robotic architecture, we do not want to make this assumption. Even if our contribution is focused on content determination, we aim with this contribution to make a first step in the linguistic realisation by not considering these to sub-task as being independent of one the other. We thus assume that not all the concepts in the knowledge base can be used in natural language.

\begin{table}[!ht]
\centering
\caption{Summary of the most representative contributions in the REG field regarding the six major features of the problem. The contributions are listed in chronological order to give a better overview of progress in the field.}
\label{tab:reg_ref_sumup}
\begin{tabular}{lcccccc}
\hline
\multicolumn{1}{|c|}{Contributions} & \multicolumn{1}{c|}{\begin{tabular}[c]{@{}c@{}}Domain\\ inde-\\ pendent\end{tabular}} & \multicolumn{1}{c|}{\begin{tabular}[c]{@{}c@{}}Rep.\\ Type\end{tabular}} & \multicolumn{1}{c|}{\begin{tabular}[c]{@{}c@{}}Use of\\ types\end{tabular}} & \multicolumn{1}{c|}{PO}  & \multicolumn{1}{c|}{\begin{tabular}[c]{@{}c@{}}Referring\\ to other\\ entities\end{tabular}} & \multicolumn{1}{c|}{\begin{tabular}[c]{@{}c@{}}Natural\\ language\end{tabular}} \\ [0.5ex] \hline \hline
\cite{dale_1989_cooking}            & \cellcolor{red!25} No                                                                 & \begin{tabular}[c]{@{}c@{}}Knowledge\\ base entity\end{tabular}          & \cellcolor{red!25} No                                                       & \cellcolor{red!25} No    & \cellcolor{red!25} No                                                                        & \cellcolor{red!25} No                                                           \\
\cite{dale_1992_generating}         & \cellcolor{red!25} No                                                                 & -                                                                        & -                                                                           & \cellcolor{red!25} No    & \cellcolor{red!25} No                                                                        & \cellcolor{red!25} No                                                           \\
\cite{reiter_1992_fast}             & \cellcolor{red!25} No                                                                 & \begin{tabular}[c]{@{}c@{}}attribute-\\ value pairs\end{tabular}         & \cellcolor{green!25} Yes                                                    & \cellcolor{green!25} Yes & \cellcolor{red!25} No                                                                        & \cellcolor{red!25} No                                                           \\
\cite{krahmer_2003_graph}           & \cellcolor{red!25} No                                                                 & REG graph                                                                & \cellcolor{red!25} No                                                       & \cellcolor{green!25} Yes & \cellcolor{green!25} Yes                                                                     & \cellcolor{red!25} No                                                           \\
\cite{yamakata_2004_belief}         & \cellcolor{red!25} No                                                                 & \begin{tabular}[c]{@{}c@{}}Belief\\ Network\end{tabular}                 & \cellcolor{red!25} No                                                       & \cellcolor{green!25} Yes & \cellcolor{red!25} No                                                                        & \cellcolor{red!25} No                                                           \\
\cite{ros_2010_which}               & \cellcolor{orange!25} Yes                                                             & Ontology                                                                 & \cellcolor{red!25} No                                                       & \cellcolor{green!25} Yes & \cellcolor{red!25} No                                                                        & \cellcolor{red!25} No                                                           \\
\cite{viethen_2013_graphs}          & \cellcolor{red!25} No                                                                 & REG graph                                                                & \cellcolor{red!25} No                                                       & \cellcolor{green!25} Yes & \cellcolor{green!25} Yes                                                                     & \cellcolor{red!25} No                                                           \\
\cite{williams_2017_referring}      & \cellcolor{orange!25} Yes                                                             & \begin{tabular}[c]{@{}c@{}}Distributed\\ KBs\end{tabular}                & \cellcolor{red!25} No                                                       & \cellcolor{green!25} Yes & \cellcolor{red!25} No                                                                        & \cellcolor{red!25} No                                                           \\
\cite{buisan_2020_efficient}        & \cellcolor{green!25} Yes                                                              & Ontology                                                                 & \cellcolor{green!25} Yes                                                    & \cellcolor{green!25} Yes & \cellcolor{green!25} Yes                                                                     & \cellcolor{green!25} Yes                                                       
\end{tabular}
\end{table}

To give a better overview of the progress in the REG field, the most representative contributions presented above are summarized in Table.~\ref{tab:reg_ref_sumup}. The contributions are organized chronologically and around six major features that we have mentioned throughout this section. These desired features are:
\begin{itemize}
	\item \textbf{Domain independent}: The knowledge base used by the REG must be able to be used by other components of a robotic architecture. The REG algorithm must not consider that all the knowledge represented can be used for this task.
	\item \textbf{Representation type}: The used knowledge representation must be able to be updated all along an interaction to deal with the dynamic of robotic applications.
	\item \textbf{Use of types}: The type of an entity is the minimal information to use to refer to an entity. Without type, linguistic realisation can not be done.
	\item \textbf{Preference ordering (PO)}: Some properties are easier to understand than others. Ordering the properties according to this preference allows finding efficient referring expressions.
	\item \textbf{Referring to other entities}: Entities attributes are not sufficient to find referring expressions in realistic situations. Being able to refer to an entity by referring to another one is thus mandatory.
	\item \textbf{Natural language}: Considering the linguistic realisation during the content generation could prevent the appearance of ambiguity at the linguistic realisation or even the incapacity to perform it.
\end{itemize}

The literature presented here before is focused on Referring Expression Generation in its nominal form. Some researches have however addressed side problems that we do not aim to tackle. Not entering much in the details, we mention them to give a more global picture of the field. The use of spatial relations is not trivial as these relations can differ for certain entities taking the RE emitter's or receiver's point of view while for other entities, having a clear orientation system (e.g. a car), the relations remain unchanged~\cite{kelleher_2006_incremental, dos_2015_generating}. Spatial relations can also be expressed not only based on a single entity but also according to a set of entity~\cite{fang_2013_towards}. While a RE is often considered as being a single sentence referring to an entity without ambiguity, some see it as a more collaborative task where the RE is provided step by step, allowing to catch acknowledgement and to refine it according to the receiver comprehension~\cite{fang_2014_collaborative, wallbridge_2019_generating}. Finally, some research tries to integrate REG in a more global interaction where several agents refer to entities. The robot thus tries to reuse properties previously used by the partner ensuring that these properties are known~\cite{williams_2020_toward}. Limitations about this work will be discussed later in this thesis.

\section{Define the REG problem}

\subsection{The knowledge representation}

\subsection{Contextualization and restriction for situated REG}

\subsection{Expected solution: structure and validity criteria}

\section{Uniform Cost Search REG}

\subsection{Formalisation as a search problem}

\subsection{Algorithm presentation}

\subsection{Replanning to to generate detailled explanations}

\section{Results}

\subsection{Solution analysis: The pen in the cup}

\subsection{Scaling up: The three-room apartment}

\subsection{Compraritions with other state-of-the art algorithms}

\subsubsection{The longest-first}

\subsubsection{The optimized Graph Based Algorithm}

\section{Proof of concept integration on a robotic system}

\section{Verbalazing a referring expression}

